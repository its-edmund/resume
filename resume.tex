\documentclass[letterpaper,8pt]{article}

\usepackage{changepage}
\usepackage{tabularx}
\usepackage{titlesec}
\usepackage[margin=1in]{geometry}
\usepackage{titling}
\usepackage{fancyhdr}
\usepackage[hidelinks]{hyperref}
\usepackage[usenames,dvipsnames]{color}
\usepackage[utf8]{inputenc}
\usepackage[english]{babel}
\usepackage{multicol}
\usepackage{enumitem}
\usepackage{titlesec}

\pagestyle{fancy}
\fancyhf{}
\fancyfoot{}
\renewcommand{\headrulewidth}{0pt}
\renewcommand{\footrulewidth}{0pt}

% Adjust margins
\addtolength{\oddsidemargin}{-0.5in}
\addtolength{\evensidemargin}{-0.5in}
\addtolength{\textwidth}{1in}
\addtolength{\topmargin}{-.5in}
\addtolength{\textheight}{1.0in}

\urlstyle{same}

\raggedbottom
\raggedright
\setlength{\tabcolsep}{0in}

% Sections formatting
\titleformat{\section}{
  \vspace{-8pt}\scshape\raggedright\large
}{}{0em}{}[\color{black}\titlerule \vspace{-6pt}]

\titleformat{\subsection}
{\vspace{-1em}\large\bfseries}
{}
{0em}
{}[\vspace{-0.5em}]

\titleformat{\subsubsection}[runin]
{\bfseries}
{}
{0em}
{}

\titlespacing{\subsubsection}
{2em}{0em}{0em}

% Custom Commands
\renewcommand{\maketitle}{
  \begin{center}
  {\huge\bfseries
  \theauthor}

  \vspace{0.25em}

  \href{mailto:mxin@gatech.edu}{mxin@gatech.edu} --- \href{https://edmundxin.me}{edmundxin.me}

  \end{center}
}

\newcommand{\resumeSubheading}[4]{
  \vspace{0.25em}
  \begin{tabular*}{0.97\textwidth}[t]{l@{\extracolsep{\fill}}r}
    \textbf{#1} & #2 \\
    \textit{#3} & \textit{#4} \\
  \end{tabular*}\vspace{0.25em}
}

\newcommand{\resumeSubheadingTwo}[2]{
  \vspace{0.25em}
  \begin{tabular*}{0.97\textwidth}[t]{l@{\extracolsep{\fill}}r}
    \textbf{#1} & #2 \\
  \end{tabular*}\vspace{0.25em}
}

\newcommand{\resumeEmphasis}[1]{
  {\small\textit{#1}\par}
}

\newcommand{\resumeSubHeadingListStart}{\begin{itemize}[leftmargin=*]}
\newcommand{\resumeSubHeadingListEnd}{\end{itemize}}
\newcommand{\resumeItemListStart}{\begin{itemize}}
\newcommand{\resumeItemListEnd}{\end{itemize}\vspace{-5pt}}


\begin{document}

\title{R\'esum\'e}
\author{Edmund Xin}

\maketitle

\section{Education}

\resumeSubheading{Georgia Institute of Technology}
{Atlanta, GA}{Bachelor of Science in Computer Science
}{Aug 2020-May 2024}
Relevant Coursework:
\setlist{nolistsep}
\begin{itemize}[label=\raisebox{0.25ex}{\tiny$\bullet$}, font=\small]
  \item CS1331 Introduction to Object Orientated Programming
\end{itemize}


\resumeSubheading{Lynbrook High School}
{San Jose, CA}{High School Diploma, U/W GPA: 3.98}{Aug 2016-Jun 2020}
Relevant Coursework:
\setlist{nolistsep}
\begin{itemize}[label=\raisebox{0.25ex}{\tiny$\bullet$}]
  \item AP Computer Science A
  \item AP Physics C: Mechanics
  \item AP Calculus BC
\end{itemize}


\section{Experience}

\resumeSubheading{Project Beck}{San Jose, CA}{Software Engineer Intern}{Jul. 2020-Present}

Student run startup focused on developing a mobile application to address mental health

\setlist{nolistsep}
\begin{itemize}[label=\raisebox{0.25ex}{\tiny$\bullet$}]
  \item Developed using React Native for mobile application
  \item Utilizes PHQ-9 questionnaire to quantify depression symptoms
\end{itemize}

\resumeSubheading{COSMOS, Machine Learning Cluster}
{University of California, Santa Cruz}{Student}{Jun. 2019-Aug. 2019}

Developed an machine learning algorithm to categorize tweets into whether or not it was relevant to a natural disaster

\setlist{nolistsep}
\begin{itemize}[label=\raisebox{0.25ex}{\tiny$\bullet$}]
  \item Used a Naive Bayes Logarithmic regressions as model for categorization
  \item Python with libaries like nltk and SciKitLearn were primarily used
  \item Tested with an 81\% accuracy in categorizing tweets
\end{itemize}

\section{Projects}

\resumeSubheadingTwo{Group Chat Analyzer}{May 2020}

Group Chat Analyzer parses and displays information from JSON data downloaded from Facebook Messenger

\setlist{nolistsep}
\begin{itemize}[label=\raisebox{0.25ex}{\tiny$\bullet$}]
  \item Used Material-UI design system for styling
  \item Data visualization using ChartJS and table to display data
\end{itemize}

\resumeSubheadingTwo{Weatherboy}
{Jun. 2020}

Weatherboy is a weather application that has a dynamic background which matches with city searched

\setlist{nolistsep}
\begin{itemize}[label=\raisebox{0.25ex}{\tiny$\bullet$}]
  \item Weekly forecast and currently temperature requested from OpenWeatherAPI
  \item Splash API used to get background with city searched as a keyword.
  \item React with styled-components to utilize transparent components
\end{itemize}


\section{Leadership}

\resumeSubheading{Lynbrook SMYLE}
{San Jose, CA}{Student Mentor}{Apr. 2019-Nov. 2019}

Taught a classes of 6 elementary students basic computer science concepts with Python and Scratch Programming Language

\resumeSubheading{Boy Scouts of America}
{San Jose, CA}{Eagle Scout}{Apr. 2018-Sep. 2018}

Earned the rank of Eagle Scout by performing a service project of building picnic tables for the safety of patrons at the Central Church of the Nazarene.

\section{Skills}

\setlist{nolistsep}
\begin{multicols}{2}
\subsection{Technical Skills}

\subsubsection{Languages}

\begin{adjustwidth}{4em}{}
Python 3, Java, C++, JavaScript(ES6), HTML, CSS, {\LaTeX}
\end{adjustwidth}

\subsubsection{Technologies}

\begin{adjustwidth}{4em}{}
ReactJS, Material-UI, Redux, Bootstrap 4, NodeJS
\end{adjustwidth}

\columnbreak

\setlist{nolistsep}
\subsection{Soft Skills}

\begin{adjustwidth}{2em}{}
Bilingual Communicator(English, Mandarin), Leadership, Teamwork, Adaptbility
\end{adjustwidth}

\end{multicols}

\end{document}
